\documentclass[a4paper]{scrartcl}
\usepackage[utf8]{inputenc} % this is needed for umlauts
\usepackage[ngerman]{babel} % this is needed for umlauts
\usepackage[T1]{fontenc}    % this is needed for correct output of umlauts in pdf
\usepackage{lmodern}
\usepackage{hyperref}   % links im text
\usepackage{csquotes}
\usepackage{titlesec}

\DeclareUrlCommand\email{\urlstyle{rm}}
\titleformat{\section}
  {\normalfont\Large\bfseries}{\S\thesection}{1em}{}

\newcommand\Gruendungsdatum{15.~Oktober~2015}
\newcommand\GruppenName{Machine Learning Karlsruhe}

\hypersetup {
  pdfauthor   = {Johannes Reiß, Marvin Schweizer, Matthias Stu, Marvin Teichmann, Martin Thoma},
  pdfkeywords = {Satzung der \GruppenName{} Hochschulgruppe},
  pdftitle    = {Satzung der \GruppenName{} Hochschulgruppe}
}

\usepackage{lastpage}
\usepackage{fancyhdr}
\fancyhf{}
\cfoot{\thepage\ von \pageref{LastPage}}
\pagestyle{fancy}
\makeatletter
\let\ps@plain\ps@fancy
\makeatother
\renewcommand{\headrulewidth}{0 mm}

\usepackage{microtype}
\begin{document}
 \author{}
\title{Satzung der Hochschulgruppe \\ \enquote{\GruppenName{}}}
\date{\today}
\maketitle


\section{Name}
Der Name der Hochschulgruppe lautet \enquote{\GruppenName{}}. Sie hat ihren
Sitz am Karlsruher Institut für Technologie (KIT).

\section{Zweck des Vereins}
\begin{enumerate}
    \item Der Zweck des Vereins ist die Förderung und Verbreitung von Wissen
          im Bereich des maschinellen Lernens.
    \item Der Satzungszweck wird insbesondere durch regelmäßige
          Zusammenkunft der aktiven Mitglieder sowie der Organisation und
          Ausführung von Veranstaltungen rund um das maschinelle Lernen
          verwirklicht.
\end{enumerate}

\section{Selbstlosigkeit}
\begin{enumerate}
    \item Die Hochschulgruppe ist selbstlos tätig und verfolgt nicht in erster
          Linie eigenwirtschaftliche Zwecke.
    \item Die Mitglieder des Vereins betätigen sich auf ehrenamtlicher Basis
          für den Verein.
\end{enumerate}

\section{Steuerbegünstigung}
\begin{enumerate}
    \item Der Verein verfolgt ausschließlich und unmittelbar gemeinnützige
	  Zwecke im Sinne des Abschnitts ,,Steuerbegünstigte Zwecke''
	  der Abgabenordnung. Der Verein ist selbstlos tätig;
	  er verfolgt nicht in erster Linie eigenwirtschaftliche Zwecke.
    \item Mittel des Vereins dürfen nur für die satzungsmäßigen Zwecke
	  verwendet werden. Die Mitglieder erhalten in ihrer Eigenschaft
	  als Mitglied keine Zuwendungen aus Mitteln des Vereins.
	  Sie haben bei ihrem Ausscheiden keinerlei Ansprüche an das
	  Vereinsvermögen. Keine Person darf durch Ausgaben, die den
	  Zwecken des Vereins fremd sind, oder durch unverhältnismäßig
	  hohe Vergütungen begünstigt werden.
\end{enumerate}

\section{Mitgliedschaft}
\begin{enumerate}
    \item Mitglied der Hochschulgruppe kann jeder am maschinellen Lernen
          Interessierte werden, insbesondere Studenten des Karlsruher
          Institutes für Technologie (KIT).
    \item Der Austritt aus der Hochschulgruppe ist jederzeit zulässig. Es
          genügt eine formlose schriftliche Mitteilung gegenüber dem Vorstand.
    \item Ein Mitglied kann aus der Hochschulgruppe ausgeschlossen werden, wenn
          sein Verhalten in grober Weise gegen die Interessen der
          Hochschulgruppe verstößt. Über den Ausschluss entscheidet der
          Vorstand. Gegen einen Ausschluss kann Widerspruch in Textform
          eingelegt werden. Über diesen Widerspruch entscheidet die nächste
          Mitgliederversammlung.
    \item Bei einer Änderung des Studentenstatus endet die Mitgliedschaft; ein
          Wiedereintritt ist in diesem Fall jederzeit möglich. Der Vorstand
          kann Mitglieder ausschließen, die nicht kontaktierbar sind und somit
          keine Auskunft über ihren Studentenstatus geben können.

\end{enumerate}

\section{Beiträge}
Von den Mitgliedern können Beiträge erhoben werden. Die Höhe der Beiträge und
deren Fälligkeit bestimmt die Mitgliederversammlung.

\section{Vorstand}
\begin{enumerate}
    \item Der Vorstand der Hochschulgruppe wird auf der Mitgliederversammlung
          gewählt. Der Vorstand ist für die Repräsentation der Hochschulgruppe
          nach außen sowie für die internen Belange der Hochschulgruppe
          verantwortlich. Der Vorstand verwaltet auch die Mitgliederliste der
          Hochschulgruppe und ist für die Protokollierung und den
          Schriftverkehr innerhalb der Hochschulgruppe zuständig.
    \item Der Vorstand wird für ein Jahr gewählt; er besteht aus drei
          Mitgliedern. Der Vorstand wählt aus seinem Kreis einen Vorsitzenden
          und einen Kassenwart. Diese Wahl ist zu protokollieren und den
          Mitgliedern zugänglich zu machen.
    \item Die Mitgliederversammlung kann den Vorstand absetzen und einen neuen
          Übergangsvorstand wählen. Dieser bleibt bis zur nächsten regulären Vorstandswahl im Amt.
\end{enumerate}



\section{Mitgliederversammlung}
\begin{enumerate}
    \item Die Mitglieder der Hochschulgruppe treffen sich zweimal jährlich zur
          Mitgliederversammlung. Auf der ersten Mitgliederversammlung im
          Wintersemester wird der Vorstand entlastet und ein neuer Vorstand
          gewählt. Zudem können auf allen Mitgliederversammlungen
          Satzungsänderungen mit einer 2/3 Mehrheit beschlossen werden.
    \item Die Mitgliederversammlung wird per E-Mail über den Verteiler
          \email{ml@lists.kit.edu} durch den Vorstand unter Einhaltung einer
          Einladungsfrist von mindestens zwei Wochen einberufen. Dabei ist die
          Tagesordnung mitzuteilen.
    \item Außerordentliche Mitgliederversammlungen können von mindestens
          fünf~Mitgliedern einberufen werden.
\end{enumerate}

\section{Vereinssitzung}
Die Vereinssitzung wird vom Vorstand einberufen, setzt sich aus den
ordentlichen Mitgliedern des Vereins zusammen und tagt regelmäßig. Sie
diskutiert und koordiniert die Aktivitäten des Vereins.

\section{Abstimmungen}
\begin{enumerate}
    \item Personenwahlen sind grundsätzlich geheim durchzuführen.
    \item Alle anderen Wahlen sind grundsätzlich öffentlich durchzuführen,
          können auf Antrag jedoch auch in geheimer Wahl durchgeführt werden.
    \item Entscheidungen werden grundsätzlich mit absoluter Mehrheit der
          anwesenden Mitglieder getroffen. Kommt keine absolute Mehrheit
          zustande ist eine Stichwahl durchzuführen.
\end{enumerate}

\section{Kassenprüfer}
\begin{enumerate}
    \item Um die sachgerechte und wirtschaftliche Verwendung der Mittel des
          Vereins zu überprüfen, bestellt eine Mitgliederversammlung zwei
          Kassenprüfer. Diese prüfen auch den Jahresabschluss.
    \item Zur Wahrnehmung ihrer Aufgaben können sie vom Vorstand alle
          erforderlichen Auskünfte mündlich und/oder schriftlich und die
          Einsicht in alle Unterlagen verlangen. Sie erstatten jeder
          ordentlichen Mitgliederversammlung einen Bericht.
    \item Zur Wahrung der Objektivität dürfen Vorstandsmitglieder nicht als
          Kassenprüfer gewählt werden.
\end{enumerate}

\section{Protokolle}
Die während der Vereinssitzung und der Mitgliederversammlung gefassten
Beschlüsse sind schriftlich niederzulegen und den Vereinsmitgliedern
zugänglich zu machen.

\section{Vereinsauflösung}
Bei Auflösung oder Aufhebung des Vereins fällt das Vermögen
des Vereins an eine juristische Person des öffentlichen Rechts
oder eine andere steuerbegünstigte Körperschaft zwecks Verwendung
für die Förderung und Verbreitung von Wissen im Bereich
des maschinellen Lernens.

\section{Inkrafttreten}
Die Satzung ist auf der Gründungsversammlung am \Gruendungsdatum{} in Karlsruhe
beschlossen worden. Sie tritt mit diesem Tage in Kraft.

% Karlsruhe, den \Gruendungsdatum{}
\end{document}
