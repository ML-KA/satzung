\documentclass[a4paper, 11pt, parskip=no, headings=small]{scrartcl}
\usepackage[utf8]{inputenc}
\usepackage[T1]{fontenc}

\usepackage[ngerman]{babel}
\usepackage{tabularx, multirow, float}

%\usepackage{amsmath,amssymb,amsthm,amsfonts,amsbsy,latexsym}
%opening
\title{Satzung der Hochschulgruppe \\ \qq{Forum Maschinelles Lernen Karlsruhe}}
\author{}
\date{\today}
\newcommand{\qq}[1]{\glqq{}#1\grqq{}}
\renewcommand{\thesection}{\S\arabic{section}}
\begin{document}
	
\maketitle

\section{Name}
Der Name der Hochschulgruppe lautet \qq{Forum Maschinelles Lernen Karlsruhe}. Sie hat ihren Sitz am Karlsruher Institut für Technologie (KIT).

\section{Zweck}
Zweck der Hochschulgruppe ist es, eine Austauschplattform über Themen und Projekte im Bereich des Maschinellen Lernens zu bieten und wissenschaftliche Diskussion und Weiterbildung in diesem Bereich zu fördern. Zur Verwirklichung dieser Ziele organisiert die Mitglieder der Hochschulgruppe regelmäßig Vortragsreihen, Lesekreise sowie praktische Projekte. 

\section{Veranstaltungen und Projekte}

Jedes Mitglied kann Veranstaltungen und Projekte mit Bezug zu Maschinellem Lernen organisieren und über die von der Hochschulgruppe bereitgestellten Kommunikationswege bewerben. Die Teilnahme an allen Veranstaltungen ist freiwillig. Abwesenheit ist kein Ausschlusskriterium aus der Hochschulgruppe. Einzelne Projekte können eine Regelmäßige Anwesenheit bei Besprechungen und aktive Beteiligung der Projektteilnehmer erfordern. Dies ist bei der Ankündigung der Ankündigung und Vorbesprechung des Projektes zu erwähnen. Ein Ausschluss von Mitgliedern aus einzelnen Projekten ist möglich.

\section{Mitgliedschaft}
Mitglied der Hochschulgruppe kann jeder interessierte Student werden, der am Karlsruher Institut für Technologie (KIT), der Karlsruher Hochschule für Musik, der Karlsruher Hochschule für Gestaltung, der Pädagogischen Hochschule Karlsruhe (PH), der Karlshochschule, der Hochschule Karlsruhe – Technik und Wirtschaft (HSKa) oder der Dualen Hochschule Baden-Württemberg Karlsruhe (DHBW Karlsruhe) immatrikuliert ist. Alle Mitglieder arbeiten ehrenamtlich.
\section{Austritt}
Der Austritt aus der Hochschulgruppe ist jederzeit zulässig. Es reicht eine formlose
Mitteilung gegenüber dem Vorstand. Ein Mitglied kann aus der Hochschulgruppe
ausgeschlossen werden, wenn sein Verhalten in grober Weise gegen die Interessen
der Hochschulgruppe verstößt.
\section{Beitrag}
Es wird kein Jahresbeitrag erhoben.
\section{Vorstand}
Der Vorstand der Hochschulgruppe wird auf der Hauptversammlung gewählt. Der Vorstand ist für die Repräsentation der Hochschulgruppe nach außen sowie für die
internen Belange der Hochschulgruppe verantwortlich. Die Mitglieder der
Hochschulgruppe wählen zudem einen Kassenwart. Der Vorstand verwalten auch die Mitgliederliste der Hochschulgruppe und sind für die Protokollierung und den Schriftverkehr innerhalb der Hochschulgruppe zuständig. Der Vorstand wird für ein Jahr gewählt.
\section{Hauptversammlung}
Die Mitglieder der Hochschulgruppe treffen sich einmal jährlich zur Hauptversammlung. Auf dieser wird der Vorstand entlastet und ein neuer Vorstand gewählt. Zudem können Sitzungsänderungen mit einer 2/3 Mehrheit beschlossen werden.
\section{Einberufung Hauptversammlung}
Die Mitgliederversammlung wird per E-Mail durch den Vorstand unter Einhaltung
einer Einladungsfrist von mindestens zwei Wochen einberufen. Dabei ist die Tages-
Ordnung mitzuteilen.

\end{document}
