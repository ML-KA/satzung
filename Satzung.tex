\documentclass[a4paper]{scrartcl}
\usepackage[utf8]{inputenc} % this is needed for umlauts
\usepackage[ngerman]{babel} % this is needed for umlauts
\usepackage[T1]{fontenc}    % this is needed for correct output of umlauts in pdf
\usepackage{hyperref}   % links im text
\usepackage{csquotes}
\usepackage{titlesec}

\titleformat{\section}
  {\normalfont\Large\bfseries}{\S\thesection}{1em}{}

\newcommand\Gruendungsdatum{10.~Oktober~2015}
\newcommand\GruppenName{ML-KIT}

\hypersetup {
  pdfauthor   = {Martin Thoma},
  pdfkeywords = {Satzung der \GruppenName{} Hochschulgruppe},
  pdftitle    = {Satzung der \GruppenName{} Hochschulgruppe}
}


\begin{document}
 \author{Martin Thoma}
 \title{Satzung der \GruppenName{} Hochschulgruppe}
 \date{27. Juli 2015}

 \maketitle

\section{Name, Sitz des Vereins}
\begin{enumerate}
    \item Die Hochschulgruppe führt den Namen \enquote{\GruppenName{}} und der Sitz ist
          Karlsruhe, Deutschland.
\end{enumerate}

\section{Zweck des Vereins}
\begin{enumerate}
    \item Der Zweck des Vereins ist die Förderung und Verbreitung von Wissen
          im Bereich des maschinellen Lernens.
    \item Der Satzungszweck wird insbesondere verwirklicht durch regelmäßige
          Zusammenkunft der aktiven Mitglieder sowie der Organisation und
          Ausführung von Veranstaltungen rund um das maschinelle Lernen.
\end{enumerate}


\section{Selbstlosigkeit}
\begin{enumerate}
    \item Die Hochschulgruppe ist selbstlos tätig und verfolgt nicht in erster
          Linie eigenwirtschaftliche Zwecke.
    \item Mittel des Vereins dürfen nur für die satzungsmäßigen Zwecke
          verwendet werden. Die Mitglieder erhalten keine Zuwendungen aus den
          Mitteln des Vereins. Es darf keine Person durch Ausgaben, die dem
          Zweck des Vereins fremd sind, oder durch unverhältnismäßig hohe
          Vergütungen begünstigt werden.
\end{enumerate}


\section{Mitgliedschaft}
\begin{enumerate}
    \item Mitglieder des Vereins können natürliche Personen werden, welche die
          Ziele des Vereins unterstützen.
    \item Die Mitgliedschaft endet, abgesehen von Tod oder Auflösung, durch:
    \begin{enumerate}
        \item Austrittserklärung in Textform seitens des Mitglieds.
        \item Streichung aus der Mitgliederliste. Die Streichung eines
              Mitglieds aus der Mitgliederliste erfolgt durch den Vorstand,
              wenn das Mitglied mit der Zahlung des Mitgliedsbeitrags mehr als
              einen Monat im Verzug ist und diesen trotz zweimaliger Mahnung,
              die in Textform zu erfolgen hat, nicht gezahlt hat. Sollten dem
              Vorstand keine aktuellen Kontaktdaten vorliegen kann die
              Streichung auch ohne Mahnung erfolgen.
        \item Ausschluss. Ein Mitglied wird aus dem Verein ausgeschlossen, wenn
              es in erheblichem Maße die Ziele des Vereins verletzt. Über den
              Ausschluss entscheidet der Vorstand. Gegen einen Ausschluss kann
              Widerspruch in Textform eingelegt werden. Über diesen Widerspruch
              entscheidet die nächste Mitgliederversammlung.
    \end{enumerate}
\end{enumerate}


\section{Beiträge}
\begin{enumerate}
    \item Von den Mitgliedern können Beiträge erhoben werden. Die Höhe der
          Beiträge und deren Fälligkeit bestimmt die Mitgliederversammlung.
\end{enumerate}


\section{Organe des Vereins}
\begin{enumerate}
    \item Die Organe des Vereins sind der Vorstand, die Vereinssitzung, die
          Kassenprüfer und die Mitgliederversammlung.
\end{enumerate}


\section{Vorstand}
\begin{enumerate}
    \item Der Vorstand setzt sich zusammen aus den nach Aufgabenbereichen
          getrennten Positionen:
          \begin{itemize}
              \item Vorsitz
              \item Finanzen
              \item Marketing
              \item Events
          \end{itemize}
          Die Vereinigung von zwei Positionen in einer Person ist unzulässig.
          Jedes Vorstandsmitglied ist einzeln zur Vertretung des Vereins
          berechtigt.
    \item Der Vorstand wird durch die Mitgliederversammlung für die Dauer von
          6~Monaten gewählt.
    \item Eine Wiederwahl ist zulässig.
    \item Jedes Mitglied des Vorstandes kann allein über Beträge bis zu einer
          Höhe von B1 pro Monat frei verfügen. Der Vorstand kann mit einer 3/4
          Mehrheit über einen Betrag bis zu einer Höhe von B2 pro Monat frei
          verfügen. Die Beträge B1 und B2 werden von der Mitgliederversammlung
          festgelegt.
\end{enumerate}


\section{Vereinssitzung}
\begin{enumerate}
    \item Die Vereinssitzung wird vom Vorstand einberufen, setzt sich aus den
          ordentlichen Mitgliedern des Vereins zusammen und tagt regelmäßig,
          mindestens jedoch einmal pro Monat. Sie diskutiert und koordiniert
          die Aktivitäten des Vereins.
    \item Die Vereinssitzung entscheidet über die Verfügung des Vorstands von
          Beträgen, die über die Beträge B1 und B2 hinausgehen.
    \item Entscheidungen werden mit einfacher Mehrheit der Anwesenden getroffen
          und finden nach demokratischen Grundsätzen statt.
\end{enumerate}


\section{Kassenprüfer}
\begin{enumerate}
    \item Um die sachgerechte und wirtschaftliche Verwendung der Mittel des
          Vereins zu überprüfen, bestellt die Mitgliederversammlung für ein
          Semester zwei Kassenprüfer. Diese prüfen auch den Semesterabschluss.
    \item Zur Wahrnehmung ihrer Aufgaben können sie vom Vorstand alle
          erforderlichen Auskünfte mündlich und/oder schriftlich und die
          Einsicht in alle Unterlagen verlangen. Sie erstatten jeder
          ordentlichen Mitgliederversammlung einen Bericht.
    \item Zur Wahrung der Objektivität dürfen Vorstandsmitglieder nicht als
          Kassenprüfer gewählt werden.
\end{enumerate}


\section{Mitgliederversammlung}
\begin{enumerate}
    \item Die Mitgliederversammlung wird vom Vorstand geleitet und findet
          einmal pro Semester statt. Jedes Mitglied kann und soll an der
          Mitgliederversammlung teilhaben und die Versammlung durch eigene Beiträge an
          der Tagesordnung ergänzen.
    \item Zur Mitgliederversammlung wird von einem Vorstandsmitglied unter
          Angabe derTagesordnung mindestens zwei Wochen vorher eingeladen.
          Eingeladen wird per E-Mail, aufschriftlichen Antrag auch auf dem
          Postweg. Satzungsänderungen sind mit der Tagesordnunganzukündigen.
    \item Die Mitgliederversammlung ist beschlussfähig, wenn sie ordnungsgemäß
          einberufen wurde.
    \item Mitglieder haben die Möglichkeit ihre Stimme (Zustimmung oder
          Ablehnung) zu Beschlüssen 3~Tage vorher in Textform an den Vorstand
          abzugeben. Nur für diese Beschlüsse zählt das Mitglied als anwesend.
    \item Beschlüsse werden mit der einfachen Mehrheit der anwesenden Mitglieder gefasst.
          Satzungsänderungen, eine Änderung des Vereinszwecks sowie eine
          Auflösung des Vereins bedürfen einer 2/3 Mehrheit der anwesenden
          Mitglieder. Mitglieder, die sich der Stimme enthalten, werden
          behandelt wie nicht erschienene.
    \item Aufgaben der Mitgliederversammlung:
        \begin{enumerate}
            \item Wahl, Abberufung und Entlastung des Vorstands
            \item Wahl der Kassenprüfer
            \item Entgegennahme des Semesterberichts des Vorstands
            \item Satzungsänderungen, Änderungen des Vereinszwecks und
                  Auflösung des Vereins
            \item Festlegen der Beträge B1, B2
        \end{enumerate}
    \item Der Vorstand hat eine außerordentliche Mitgliederversammlung
          einzuberufen, wenn dies im Interesse des Vereins erforderlich ist
          oder die Einberufung durch 1/10 der Mitglieder schriftlich unter
          Angabe von Gründen verlangt wird.
\end{enumerate}

\section{Protokolle}
\begin{enumerate}
    \item Die während der Vereinssitzung und der Mitgliederversammlung
          gefassten Beschlüsse sind schriftlich niederzulegen und den
          Vereinsmitgliedern zugänglich zu machen.
\end{enumerate}


\section{Inkrafttreten}
\begin{enumerate}
    \item Die Satzung ist auf der Gründungsversammlung am \Gruendungsdatum{} in
          Karlsruhe beschlossen worden. Sie tritt mit diesem Tage in Kraft.
\end{enumerate}


Karlsruhe, den \Gruendungsdatum{}

\end{document}
